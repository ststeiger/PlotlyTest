%% ================================================================================
%% This LaTeX file was created by AbiWord.                                         
%% AbiWord is a free, Open Source word processor.                                  
%% More information about AbiWord is available at http://www.abisource.com/        
%% ================================================================================

\documentclass[a4paper,portrait,12pt]{article}
\usepackage[latin1]{inputenc}
\usepackage{calc}
\usepackage{setspace}
\usepackage{fixltx2e}
\usepackage{graphicx}
\usepackage{multicol}
\usepackage[normalem]{ulem}
%% Please revise the following command, if your babel
%% package does not support en-US
\usepackage[en]{babel}
\usepackage{color}
\usepackage{hyperref}
 
\begin{document}


\begin{flushleft}
arXiv:1412.7690v1 [physics.geo-ph] 16 Dec 2014
\end{flushleft}





\begin{flushleft}
Map Projection
\end{flushleft}


\begin{flushleft}
Ebrahim Ghaderpour
\end{flushleft}


\begin{flushleft}
Email: ebig2@yorku.ca
\end{flushleft}


\begin{flushleft}
Department of Earth and Space Science and Engineering,
\end{flushleft}


\begin{flushleft}
York University, Toronto, Canada
\end{flushleft}





\begin{flushleft}
Abstract
\end{flushleft}


\begin{flushleft}
In this paper, we introduce some known map projections from a model of the Earth to a flat sheet
\end{flushleft}


\begin{flushleft}
of paper or map and derive the plotting equations for these projections. The first fundamental form
\end{flushleft}


\begin{flushleft}
and the Gaussian fundamental quantities are defined and applied to obtain the plotting equations
\end{flushleft}


\begin{flushleft}
and distortions in length, shape and size for some of these map projections.
\end{flushleft}





\begin{flushleft}
The concepts, definitions and proofs in this work are chosen mostly from [5, 6].
\end{flushleft}





1





\begin{flushleft}
Introduction
\end{flushleft}





\begin{flushleft}
A map projection is a systematic transformation of the latitudes and longitudes of positions on the surface
\end{flushleft}


\begin{flushleft}
of the Earth to a flat sheet of paper, a map. More precisely, a map projection requires a transformation
\end{flushleft}


\begin{flushleft}
from a set of two independent coordinates on the model of the Earth (the latitude $\phi$ and longitude $\lambda$) to a
\end{flushleft}


\begin{flushleft}
set of two independent coordinates on the map (the Cartesian coordinates x and y), i.e., a transformation
\end{flushleft}


\begin{flushleft}
matrix T such that
\end{flushleft}


?


?


?


?


\begin{flushleft}
x
\end{flushleft}


\begin{flushleft}
$\phi$
\end{flushleft}


\begin{flushleft}
=T
\end{flushleft}


.


\begin{flushleft}
y
\end{flushleft}


\begin{flushleft}
$\lambda$
\end{flushleft}


\begin{flushleft}
However, since we are dealing with partial derivative and fundamental quantities (to be defined later), it
\end{flushleft}


\begin{flushleft}
is impossible to find such a transformation explicitly.
\end{flushleft}


\begin{flushleft}
There are a number of techniques for map projection, yet in all of them distortion occurs in length,
\end{flushleft}


\begin{flushleft}
angle, shape, area or in a combination of these. Carl Friedrich Gauss showed that a sphere's surface
\end{flushleft}


\begin{flushleft}
cannot be represented on a map without distortion (see [5]).
\end{flushleft}


\begin{flushleft}
A terrestrial globe is a three dimensional scale model of the Earth that does not distort the real
\end{flushleft}


\begin{flushleft}
shape and the real size of large futures of the Earth. The term globe is used for those objects that are
\end{flushleft}


\begin{flushleft}
approximately spherical. The equation for spherical model of the Earth with radius R is
\end{flushleft}


\begin{flushleft}
x2
\end{flushleft}


\begin{flushleft}
y2
\end{flushleft}


\begin{flushleft}
z2
\end{flushleft}


+


+


= 1.


\begin{flushleft}
R2
\end{flushleft}


\begin{flushleft}
R2
\end{flushleft}


\begin{flushleft}
R2
\end{flushleft}





(1)





\begin{flushleft}
An oblate ellipsoid or spheroid is a quadratic surface obtained by rotating an ellipse about its minor
\end{flushleft}


\begin{flushleft}
axis (the axis that passes through the north pole and the south pole). The shape of the Earth is appeared
\end{flushleft}


\begin{flushleft}
to be an oblate ellipsoid (mean Earth ellipsoid), and the geodetic latitudes and longitudes of positions
\end{flushleft}


\begin{flushleft}
on the surface of the Earth coming from satellite observations are on this ellipsoid. The equation for
\end{flushleft}





1





\begin{flushleft}
\newpage
spheroidal model of the Earth is
\end{flushleft}


\begin{flushleft}
x2
\end{flushleft}


\begin{flushleft}
y2
\end{flushleft}


\begin{flushleft}
z2
\end{flushleft}


+ 2 + 2 = 1,


2


\begin{flushleft}
a
\end{flushleft}


\begin{flushleft}
a
\end{flushleft}


\begin{flushleft}
b
\end{flushleft}





(2)





\begin{flushleft}
where a is the semimajor axis, and b is the semiminor axis of the spheroid of revolution.
\end{flushleft}


\begin{flushleft}
The spherical representation of the Earth (terrestrial globe) must be modified to maintain accurate
\end{flushleft}


\begin{flushleft}
representation of either shape or size of the spheroidal representation of the Earth. We discuss about
\end{flushleft}


\begin{flushleft}
these two representations in Section 4.
\end{flushleft}


\begin{flushleft}
There are three major types of map projections:
\end{flushleft}


\begin{flushleft}
1. Equal-area projections. These projections preserve the area (the size) between the map and
\end{flushleft}


\begin{flushleft}
the model of the Earth. In other words, every section of the map keeps a constant ratio to the area of the
\end{flushleft}


\begin{flushleft}
Earth which it represents. Some of these projections are Albers with one or two standard parallels (the
\end{flushleft}


\begin{flushleft}
conical equal-area), the Bonne, the azimuthal and Lambert cylindrical equal-area which are best applied
\end{flushleft}


\begin{flushleft}
to a local area of the Earth, and some of them are world maps such as the sinusoidal, the Mollweide, the
\end{flushleft}


\begin{flushleft}
parabolic, the Hammer-Aitoff, the Boggs eumorphic, and Eckert IV.
\end{flushleft}


\begin{flushleft}
2. Conformal projections. These projections maintain the shape of an area during transformation
\end{flushleft}


\begin{flushleft}
from the Earth to a map. These projections include the Mercator, the Lambert conformal with one
\end{flushleft}


\begin{flushleft}
standard parallel, and the stereographic. These projections are only applicable to limited areas on the
\end{flushleft}


\begin{flushleft}
model of the Earth for any one map. Since there is no practical use for conformal world maps, conformal
\end{flushleft}


\begin{flushleft}
world maps are not considered.
\end{flushleft}


\begin{flushleft}
3. Conventional projections. These projections are neither equal-area nor conformal, and they
\end{flushleft}


\begin{flushleft}
are designed based on some particular applications. Some examples are the simple conic, the gnomonic,
\end{flushleft}


\begin{flushleft}
the azimuthal equidistant, the Miller, the polyconic, the Robinson, and the plate carree projections.
\end{flushleft}


\begin{flushleft}
In this paper, we only show the derivation of plotting equations on a map for the Mercator and
\end{flushleft}


\begin{flushleft}
Lambert cylindrical equal-area for a spherical model of the Earth (Section 2), the Albers with one
\end{flushleft}


\begin{flushleft}
standard parallel and the azimuthal for a spherical model of the Earth and the Lambert conformal with
\end{flushleft}


\begin{flushleft}
one standard parallel for a spheroidal model of the Earth (Section 5), the sinusoidal (Section 6), the
\end{flushleft}


\begin{flushleft}
simple conic and the plate carree projections (Section 7). The methods to obtain other projections are
\end{flushleft}


\begin{flushleft}
similar to these projections, and the reader is referred to [2, 5, 6].
\end{flushleft}


\begin{flushleft}
Suppose that a terrestrial glob is covered with infinitesimal circles. In order to show distortions in a
\end{flushleft}


\begin{flushleft}
map projection, one may look at the projection of these circles in a map which are ellipses whose axes are
\end{flushleft}


\begin{flushleft}
the two principal directions along which scale is maximal and minimal at that point on the map. This
\end{flushleft}


\begin{flushleft}
mathematical contrivance is called Tissot's indicatrix.
\end{flushleft}


\begin{flushleft}
Usually Tissot's indicatrices are placed across a map along the intersections of meridians and parallels
\end{flushleft}


\begin{flushleft}
to the equator, and they provide a good tool to calculate the magnitude of distortions at those points
\end{flushleft}


\begin{flushleft}
(the intersections).
\end{flushleft}


\begin{flushleft}
In an equal-area projection, Tissot's indicatrices change shape (from circles to ellipses), whereas their
\end{flushleft}


\begin{flushleft}
areas remain the same. In conformal projection, however, the shape of circles preserves, and the area
\end{flushleft}


\begin{flushleft}
varies. In conventional projection, both shape and size of these circles change. In this paper, we portrayed
\end{flushleft}


\begin{flushleft}
the Mercator, the Lambert cylindrical equal-area, the sinusoidal and the plate carree maps with Tissot's
\end{flushleft}


\begin{flushleft}
indicatrices.
\end{flushleft}


\begin{flushleft}
In Section 8, the equations for distortions of length, area and angle are derived, and distortion in length
\end{flushleft}


\begin{flushleft}
for the Albers projection and in length and area for the Mercator projection are calculated, [5, 6, 9].
\end{flushleft}





2





\newpage
2





\begin{flushleft}
Mercator projection and Lambert cylindrical projection
\end{flushleft}





\begin{flushleft}
In this section, by an elementary method, we show the cylindrical method that Mercator used to map
\end{flushleft}


\begin{flushleft}
from a spherical model of the Earth to a flat sheet of paper. Also, we give the plotting equations for
\end{flushleft}


\begin{flushleft}
the Lambert cylindrical equal-area projection. Then, in Section 3, we obtain the Gaussian fundamental
\end{flushleft}


\begin{flushleft}
quantities, and show a routine mathematical way to find plotting equations for different map projections.
\end{flushleft}


\begin{flushleft}
This section is based mostly on [4].
\end{flushleft}


\begin{flushleft}
Let S be the globe, and C be a circular cylinder tangent to S along the equator, see Fig. 1. Projecting
\end{flushleft}


\begin{flushleft}
S along the rays passing through the center of S onto C, and unrolling the cylinder onto a vertical strip
\end{flushleft}


\begin{flushleft}
in a plane is called central cylindrical projection. Clearly, each meridian on the sphere is mapped to a
\end{flushleft}


\begin{flushleft}
vertical line to the equator, and each parallel of the equator is mapped onto a circle on the cylinder and
\end{flushleft}


\begin{flushleft}
so a line parallel to the equator on the map. All methods discussed in this section and other sections are
\end{flushleft}





\begin{flushleft}
Figure 1: Geometry for the cylindrical projection
\end{flushleft}


\begin{flushleft}
about central projection, i.e., rays pass through the center of the Earth to a cone or cylinder. Methods
\end{flushleft}


\begin{flushleft}
for those projections that are not central are similar to central projections (see [5, 6]).
\end{flushleft}


\begin{flushleft}
Let w be the width of the map. The scale of the map along the equator is s = w/(2$\pi$R) that is the
\end{flushleft}


\begin{flushleft}
ratio of size of objects drawn in the map to actual size of the object it represents. The scale of the map
\end{flushleft}


\begin{flushleft}
usually is shown by three methods: arithmetical (e.g. 1:6,000,000), verbal (e.g. 100 miles to the inch) or
\end{flushleft}


\begin{flushleft}
geometrical.
\end{flushleft}


\begin{flushleft}
At latitude $\phi$, the parallel to the equator is a circle with circumference 2$\pi$R cos $\phi$, so the scale of the
\end{flushleft}


\begin{flushleft}
map at this latitude is
\end{flushleft}


\begin{flushleft}
sh =
\end{flushleft}





\begin{flushleft}
w
\end{flushleft}


\begin{flushleft}
= s sec $\phi$,
\end{flushleft}


\begin{flushleft}
2$\pi$R cos $\phi$
\end{flushleft}





(3)





\begin{flushleft}
where the subscript h stands for horizontal.
\end{flushleft}


\begin{flushleft}
Assume that $\phi$ and $\lambda$ are in radians, and the origin in the Cartesian coordinate system corresponds
\end{flushleft}


\begin{flushleft}
to the intersection of the Greenwich meridian ($\lambda$ = 0) and the equator ($\phi$ = 0). Then every cylindrical
\end{flushleft}


3





\begin{flushleft}
\newpage
projection is given explicitly by the following equations
\end{flushleft}


\begin{flushleft}
x=
\end{flushleft}





\begin{flushleft}
w$\lambda$
\end{flushleft}


,


\begin{flushleft}
2$\pi$
\end{flushleft}





\begin{flushleft}
y = f ($\phi$).
\end{flushleft}





(4)





\begin{flushleft}
For instance, it can be seen from Fig. 1 that a central cylindrical projection is given by
\end{flushleft}


\begin{flushleft}
x=
\end{flushleft}





\begin{flushleft}
w$\lambda$
\end{flushleft}


,


\begin{flushleft}
2$\pi$
\end{flushleft}





\begin{flushleft}
y = r tan $\phi$,
\end{flushleft}





\begin{flushleft}
where for a map of width w, a globe of radius r = w/(2$\pi$) is chosen.
\end{flushleft}


\begin{flushleft}
In a globe, the arc length between latitudes of $\phi$ and $\phi$1 (in radians) along a meridian is
\end{flushleft}


\begin{flushleft}
2$\pi$R ·
\end{flushleft}





\begin{flushleft}
$\phi$1 $-$ $\phi$
\end{flushleft}


\begin{flushleft}
= R($\phi$1 $-$ $\phi$),
\end{flushleft}


\begin{flushleft}
2$\pi$
\end{flushleft}





\begin{flushleft}
and the image on the map has the length f ($\phi$1 ) $-$ f ($\phi$). So the overall scale factor of this arc along the
\end{flushleft}


\begin{flushleft}
meridian when $\phi$1 gets closer and closer to $\phi$ is
\end{flushleft}


\begin{flushleft}
sv =
\end{flushleft}





1


\begin{flushleft}
f ($\phi$1 ) $-$ f ($\phi$)
\end{flushleft}


1 0


\begin{flushleft}
f ($\phi$) =
\end{flushleft}


\begin{flushleft}
lim
\end{flushleft}


,


\begin{flushleft}
R
\end{flushleft}


\begin{flushleft}
R $\phi$1 $\rightarrow$$\phi$
\end{flushleft}


\begin{flushleft}
$\phi$1 $-$ $\phi$
\end{flushleft}





(5)





\begin{flushleft}
where the subscript v stands for vertical.
\end{flushleft}


\begin{flushleft}
The goal of Mercator was to equate the horizontal scale with vertical scale at latitude $\phi$, i.e., sh = sv .
\end{flushleft}


\begin{flushleft}
Thus, from Eqs. (3) and (5),
\end{flushleft}


\begin{flushleft}
f 0 ($\phi$) =
\end{flushleft}





\begin{flushleft}
w
\end{flushleft}


\begin{flushleft}
sec $\phi$.
\end{flushleft}


\begin{flushleft}
2$\pi$
\end{flushleft}





(6)





\begin{flushleft}
Mercator was not be able to solve Equation 6 precisely because logarithms were not invented! But now,
\end{flushleft}


\begin{flushleft}
we know that the following is the solution to Eq. (6) (use f (0) = 0 to make the constant coming out
\end{flushleft}


\begin{flushleft}
from the integration equal to zero),
\end{flushleft}


\begin{flushleft}
y = f ($\phi$) =
\end{flushleft}





\begin{flushleft}
w
\end{flushleft}


\begin{flushleft}
ln | sec $\phi$ + tan $\phi$|.
\end{flushleft}


\begin{flushleft}
2$\pi$
\end{flushleft}





(7)





\begin{flushleft}
Thus, the equations for the Mercator conformal projection (central cylindrical conformal mapping) are
\end{flushleft}


\begin{flushleft}
x=
\end{flushleft}





\begin{flushleft}
w$\lambda$
\end{flushleft}


,


\begin{flushleft}
2$\pi$
\end{flushleft}





\begin{flushleft}
y=
\end{flushleft}





\begin{flushleft}
w
\end{flushleft}


\begin{flushleft}
ln | sec $\phi$ + tan $\phi$|.
\end{flushleft}


\begin{flushleft}
2$\pi$
\end{flushleft}





\begin{flushleft}
Fig. 2 shows the Mercator projection with Tissot's indicatrices that do not change their shape (all of
\end{flushleft}


\begin{flushleft}
them are circles indicating a conformal projection) while their size get larger and larger toward the poles.
\end{flushleft}


\begin{flushleft}
Now if the goal is preserving size rather than shape, then we would make the horizontal and vertical
\end{flushleft}


\begin{flushleft}
scaling reciprocal, so the stretching in one direction will match shrinking in the other. Thus, from Eqs.
\end{flushleft}


\begin{flushleft}
(3) and (5), we obtain f 0 ($\phi$) sec $\phi$ = c or
\end{flushleft}


\begin{flushleft}
f 0 ($\phi$) = c cos $\phi$,
\end{flushleft}





(8)





\begin{flushleft}
where c is a constant. From Eqs. (6) and (8), we can choose c in such away that for a given latitude, the
\end{flushleft}


\begin{flushleft}
map also preserves the shape in that area. For instance if $\phi$ = 0, then we choose c = w/(2$\pi$), and so the
\end{flushleft}


\begin{flushleft}
map near equator is conformal too. Hence, the equations for the cylindrical equal-area projection (one
\end{flushleft}


\begin{flushleft}
of Lambert's maps) are
\end{flushleft}


\begin{flushleft}
w$\lambda$
\end{flushleft}


\begin{flushleft}
w
\end{flushleft}


\begin{flushleft}
x=
\end{flushleft}


,


\begin{flushleft}
y=
\end{flushleft}


\begin{flushleft}
sin $\phi$.
\end{flushleft}


\begin{flushleft}
2$\pi$
\end{flushleft}


\begin{flushleft}
2$\pi$
\end{flushleft}


\begin{flushleft}
Fig. 3 shows the Lambert projection with Tissot's indicatrices that do not change their size (indicating
\end{flushleft}


\begin{flushleft}
an equal-area projection) while their shape are changing toward the poles.
\end{flushleft}


4





\begin{flushleft}
\newpage
Figure 2: The Mercator conformal map
\end{flushleft}





3





\begin{flushleft}
Figure 3: The Lambert equal-area map
\end{flushleft}





\begin{flushleft}
First fundamental form
\end{flushleft}





\begin{flushleft}
In this section, we derive the first fundamental form for a general surface that completely describes the
\end{flushleft}


\begin{flushleft}
metric properties of the surface, and it is a key in map projection, [3, 5, 6, 9].
\end{flushleft}


$-$


$-$


\begin{flushleft}
The vector at any point P on the surface is given by $\rightarrow$
\end{flushleft}


\begin{flushleft}
r =$\rightarrow$
\end{flushleft}


\begin{flushleft}
r ($\alpha$, $\beta$). If either of parameters $\alpha$ or $\beta$ is
\end{flushleft}


\begin{flushleft}
held constant and the other one is varied, a space curve results, see Fig. 4.
\end{flushleft}





\begin{flushleft}
Figure 4: Geometry for parametric curves
\end{flushleft}


\begin{flushleft}
The tangent vectors to $\alpha$-curve and $\beta$-curve at point P are respectively as follows:
\end{flushleft}


$-$


$-$


$\rightarrow$


$-$


$\partial$$\rightarrow$


\begin{flushleft}
r
\end{flushleft}


$\partial$$\rightarrow$


\begin{flushleft}
r
\end{flushleft}


$\rightarrow$


$-$


\begin{flushleft}
a =
\end{flushleft}


,


\begin{flushleft}
b =
\end{flushleft}


.


\begin{flushleft}
$\partial$$\alpha$
\end{flushleft}


\begin{flushleft}
$\partial$$\beta$
\end{flushleft}


$-$


\begin{flushleft}
The total differential of $\rightarrow$
\end{flushleft}


\begin{flushleft}
r is
\end{flushleft}


$\rightarrow$


$-$


$-$


$-$


\begin{flushleft}
d$\rightarrow$
\end{flushleft}


\begin{flushleft}
r =$\rightarrow$
\end{flushleft}


\begin{flushleft}
a d$\alpha$ + b d$\beta$.
\end{flushleft}


5





(9)





(10)





\begin{flushleft}
\newpage
The first fundamental form (e.g., [5]) is defined as the dot product of Eq. (10) with itself:
\end{flushleft}


$\rightarrow$


$-$ ?


$\rightarrow$


$-$ ? $-$


$-$


$-$


$-$


\begin{flushleft}
a d$\alpha$ + b d$\beta$
\end{flushleft}


\begin{flushleft}
a d$\alpha$ + b d$\beta$ · $\rightarrow$
\end{flushleft}


\begin{flushleft}
(ds)2 = d$\rightarrow$
\end{flushleft}


\begin{flushleft}
r · d$\rightarrow$
\end{flushleft}


\begin{flushleft}
r = $\rightarrow$
\end{flushleft}


\begin{flushleft}
= E(d$\alpha$)2 + 2F d$\alpha$d$\beta$ + G(d$\beta$)2 ,
\end{flushleft}





(11)





$\rightarrow$


$-$


$\rightarrow$


$-$ $\rightarrow$


$-$


$-$


$-$


$-$


\begin{flushleft}
where E = $\rightarrow$
\end{flushleft}


\begin{flushleft}
a ·$\rightarrow$
\end{flushleft}


\begin{flushleft}
a, F =$\rightarrow$
\end{flushleft}


\begin{flushleft}
a · b and G = b · b are known as the Gaussian fundamental quantities.
\end{flushleft}


\begin{flushleft}
$\bullet$ From Eq. (11), the distance between two arbitrary points P1 and P2 on the surface can be calculated:
\end{flushleft}


\begin{flushleft}
Z P2 p
\end{flushleft}


\begin{flushleft}
Z P2 r
\end{flushleft}


\begin{flushleft}
? d$\beta$ ?
\end{flushleft}


\begin{flushleft}
? d$\beta$ ?2
\end{flushleft}


2


2


\begin{flushleft}
s=
\end{flushleft}


\begin{flushleft}
E(d$\alpha$) + 2F d$\alpha$d$\beta$ + G(d$\beta$) =
\end{flushleft}


\begin{flushleft}
E + 2F
\end{flushleft}


\begin{flushleft}
+G
\end{flushleft}


\begin{flushleft}
d$\alpha$.
\end{flushleft}


\begin{flushleft}
d$\alpha$
\end{flushleft}


\begin{flushleft}
d$\alpha$
\end{flushleft}


\begin{flushleft}
P1
\end{flushleft}


\begin{flushleft}
P1
\end{flushleft}


$\rightarrow$


$-$


$-$


\begin{flushleft}
$\bullet$ The angle between $\rightarrow$
\end{flushleft}


\begin{flushleft}
a and b is simply given by
\end{flushleft}


$\rightarrow$


$-$


$\rightarrow$


$-$


\begin{flushleft}
a · b
\end{flushleft}


\begin{flushleft}
F
\end{flushleft}


\begin{flushleft}
cos $\theta$ =
\end{flushleft}


.


$\rightarrow$


$-$ =$\surd$


$\rightarrow$


$-$


\begin{flushleft}
EG
\end{flushleft}


\begin{flushleft}
| a |.| b |
\end{flushleft}





(12)





$\rightarrow$


$-$


$-$


\begin{flushleft}
$\bullet$ Incremental area is the magnitude of the cross product of $\rightarrow$
\end{flushleft}


\begin{flushleft}
a d$\alpha$ and b d$\beta$, i.e.,
\end{flushleft}


$\rightarrow$


$-$


$\rightarrow$


$-$


$-$


$-$


\begin{flushleft}
dA = |$\rightarrow$
\end{flushleft}


\begin{flushleft}
a d$\alpha$ × b d$\beta$| = |$\rightarrow$
\end{flushleft}


\begin{flushleft}
a d$\alpha$|.| b d$\beta$| sin $\theta$
\end{flushleft}


$\rightarrow$


$-$


$-$


= |$\rightarrow$


\begin{flushleft}
a |.| b | sin $\theta$d$\alpha$d$\beta$
\end{flushleft}


\begin{flushleft}
$\surd$ $\surd$ p
\end{flushleft}


\begin{flushleft}
= E G 1 $-$ cos2 $\theta$d$\alpha$d$\beta$
\end{flushleft}


\begin{flushleft}
r
\end{flushleft}


$\surd$


\begin{flushleft}
EG $-$ F 2
\end{flushleft}


\begin{flushleft}
d$\alpha$d$\beta$
\end{flushleft}


\begin{flushleft}
from Eq. (12)
\end{flushleft}


\begin{flushleft}
= EG
\end{flushleft}


\begin{flushleft}
EG
\end{flushleft}


\begin{flushleft}
p
\end{flushleft}


\begin{flushleft}
= EG $-$ F 2 d$\alpha$d$\beta$.
\end{flushleft}





(13)





\begin{flushleft}
Since we are dealing with latitudes and longitudes on a spherical or spheroidal model of the Earth,
\end{flushleft}


$\rightarrow$


$-$


$-$


\begin{flushleft}
the vectors $\rightarrow$
\end{flushleft}


\begin{flushleft}
a and b are orthogonal (meridians are normal to equator parallels). Also, in maps, we are
\end{flushleft}


\begin{flushleft}
dealing with the polar and Cartesian coordinate systems in which their axes are perpendicular. Thus,
\end{flushleft}


\begin{flushleft}
from Eq. (12), because cos 90◦ = 0, one obtains F = 0.
\end{flushleft}


\begin{flushleft}
Therefore, the first fundamental form (11) in map projection will be deduced to the following form:
\end{flushleft}


\begin{flushleft}
(ds)2 = E(d$\alpha$)2 + G(d$\beta$)2 .
\end{flushleft}





(14)





\begin{flushleft}
Example 1 The first fundamental form for a planar surface
\end{flushleft}


\begin{flushleft}
1. in the Cartesian coordinate system (a cylindrical surface) is (ds)2 = (dx)2 +(dy)2 , where E = G = 1,
\end{flushleft}


\begin{flushleft}
2. in the polar coordinate system (a conical surface) is (ds)2 = (dr)2 + r2 (d$\theta$)2 , where E = 1 and
\end{flushleft}


\begin{flushleft}
G = r2 ,
\end{flushleft}


\begin{flushleft}
3. in the spherical model of the Earth, Eq. (1), is (ds)2 = R2 (d$\phi$)2 + R2 cos2 $\phi$(d$\lambda$)2 , where E = R2
\end{flushleft}


\begin{flushleft}
and G = R2 cos2 $\phi$, and
\end{flushleft}


\begin{flushleft}
4. in the spheroidal model of the Earth, Eq. (2), is (ds)2 = M 2 (d$\phi$)2 + N 2 cos2 $\phi$(d$\lambda$)2 , where E = M 2
\end{flushleft}


\begin{flushleft}
and G = N 2 cos2 $\phi$ in which M is the radius of curvature in meridian and N is the radius of curvature in
\end{flushleft}


\begin{flushleft}
prime vertical which are both functions of $\phi$:
\end{flushleft}


\begin{flushleft}
M=
\end{flushleft}





\begin{flushleft}
a(1 $-$ e2ab )
\end{flushleft}


,


\begin{flushleft}
(1 $-$ e2ab sin2 $\phi$)1.5
\end{flushleft}





\begin{flushleft}
N=
\end{flushleft}





\begin{flushleft}
a
\end{flushleft}


,


\begin{flushleft}
(1 $-$ e2ab sin2 $\phi$)0.5
\end{flushleft}





\begin{flushleft}
See Fig. 5, and [5, 9] for the derivations of M and N .
\end{flushleft}


6





\begin{flushleft}
e2ab =
\end{flushleft}





\begin{flushleft}
a2 $-$ b2
\end{flushleft}


.


\begin{flushleft}
a2
\end{flushleft}





\begin{flushleft}
\newpage
Figure 5: Geometry for the spheroidal model of the Earth,
\end{flushleft}





\begin{flushleft}
x2
\end{flushleft}


\begin{flushleft}
y2
\end{flushleft}


\begin{flushleft}
z2
\end{flushleft}


+


+


= 1.


\begin{flushleft}
a2
\end{flushleft}


\begin{flushleft}
a2
\end{flushleft}


\begin{flushleft}
b2
\end{flushleft}





\begin{flushleft}
Now suppose that $\phi$ and $\lambda$ are the parameters of the model of the Earth with the fundamental
\end{flushleft}


\begin{flushleft}
quantities e, f and g.
\end{flushleft}


\begin{flushleft}
Consider a two-dimensional projection with parametric curves defined by the parameters u and v.
\end{flushleft}


\begin{flushleft}
For instance, for the polar or conical coordinates, we have u = r and v = $\theta$. Let E 0 , F 0 and G0 be its
\end{flushleft}


\begin{flushleft}
fundamental quantities.
\end{flushleft}


\begin{flushleft}
Also, assume that on the plotting surface a second set of parameters, x and y, with the fundamental
\end{flushleft}


\begin{flushleft}
quantities E, F and G.
\end{flushleft}


\begin{flushleft}
The relationship between the two sets of parameters on the plane is given by
\end{flushleft}


\begin{flushleft}
x = x(u, v),
\end{flushleft}





\begin{flushleft}
y = y(u, v).
\end{flushleft}





(15)





\begin{flushleft}
As an example, x = r cos $\theta$ and y = r sin $\theta$ for the polar and Cartesian coordinates.
\end{flushleft}


\begin{flushleft}
The relationship between the parametric curves $\phi$, $\lambda$, u and v is
\end{flushleft}


\begin{flushleft}
u = u($\phi$, $\lambda$),
\end{flushleft}





\begin{flushleft}
v = v($\phi$, $\lambda$).
\end{flushleft}





(16)





\begin{flushleft}
Eq. (16) must be unique and reversible, i.e., a point on the Earth must represent only one point on
\end{flushleft}


\begin{flushleft}
the map and vice versa. From Eqs. (15) and (16), we have
\end{flushleft}


?


?


\begin{flushleft}
x = x u($\phi$, $\lambda$), v($\phi$, $\lambda$) ,
\end{flushleft}


\begin{flushleft}
y = y u($\phi$, $\lambda$), v($\phi$, $\lambda$) .
\end{flushleft}


(17)


\begin{flushleft}
From the definition of the Gaussian first fundamental quantities, we have
\end{flushleft}





7





\begin{flushleft}
\newpage
$\partial$y ? ? $\partial$x
\end{flushleft}


·


,


\begin{flushleft}
$\partial$$\phi$ $\partial$$\phi$
\end{flushleft}


\begin{flushleft}
$\partial$$\phi$
\end{flushleft}


\begin{flushleft}
? $\partial$x $\partial$y ? ? $\partial$x
\end{flushleft}


$\rightarrow$


$-$


$-$


\begin{flushleft}
F =$\rightarrow$
\end{flushleft}


\begin{flushleft}
a · b =
\end{flushleft}


·


,


,


\begin{flushleft}
$\partial$$\phi$ $\partial$$\phi$
\end{flushleft}


\begin{flushleft}
$\partial$$\lambda$
\end{flushleft}


?


?


?


$\rightarrow$


$-$ $\rightarrow$


$-$


\begin{flushleft}
$\partial$x $\partial$y
\end{flushleft}


\begin{flushleft}
$\partial$x
\end{flushleft}


\begin{flushleft}
G= b · b =
\end{flushleft}


·


,


,


\begin{flushleft}
$\partial$$\lambda$ $\partial$$\lambda$
\end{flushleft}


\begin{flushleft}
$\partial$$\lambda$
\end{flushleft}





$-$


$-$


\begin{flushleft}
E=$\rightarrow$
\end{flushleft}


\begin{flushleft}
a ·$\rightarrow$
\end{flushleft}


\begin{flushleft}
a =
\end{flushleft}





\begin{flushleft}
? $\partial$x
\end{flushleft}





,





\begin{flushleft}
$\partial$y ? ? $\partial$x ?2 ? $\partial$y ?2
\end{flushleft}


=


+


,


\begin{flushleft}
$\partial$$\phi$
\end{flushleft}


\begin{flushleft}
$\partial$$\phi$
\end{flushleft}


\begin{flushleft}
$\partial$$\phi$
\end{flushleft}


\begin{flushleft}
$\partial$y ? $\partial$x $\partial$x
\end{flushleft}


\begin{flushleft}
$\partial$y $\partial$y
\end{flushleft}


=


+


,


\begin{flushleft}
$\partial$$\lambda$
\end{flushleft}


\begin{flushleft}
$\partial$$\phi$ $\partial$$\lambda$ $\partial$$\phi$ $\partial$$\lambda$
\end{flushleft}


?


?


?


?


?


\begin{flushleft}
$\partial$x 2
\end{flushleft}


\begin{flushleft}
$\partial$y
\end{flushleft}


\begin{flushleft}
$\partial$y 2
\end{flushleft}


=


+


.


\begin{flushleft}
$\partial$$\lambda$
\end{flushleft}


\begin{flushleft}
$\partial$$\lambda$
\end{flushleft}


\begin{flushleft}
$\partial$$\lambda$
\end{flushleft}





\begin{flushleft}
Note that in here $\alpha$ and $\beta$ in (9) are replaced by $\phi$ and $\lambda$, respectively. Similarly, we have
\end{flushleft}


\begin{flushleft}
? $\partial$x $\partial$y ? ? $\partial$x $\partial$y ? ? $\partial$x ?2 ? $\partial$y ?2
\end{flushleft}


$-$


$-$


·


=


\begin{flushleft}
E0 = $\rightarrow$
\end{flushleft}


\begin{flushleft}
a ·$\rightarrow$
\end{flushleft}


\begin{flushleft}
a =
\end{flushleft}


,


,


+


,


\begin{flushleft}
$\partial$u $\partial$u
\end{flushleft}


\begin{flushleft}
$\partial$u $\partial$u
\end{flushleft}


\begin{flushleft}
$\partial$u
\end{flushleft}


\begin{flushleft}
$\partial$u
\end{flushleft}


?


?


?


?


$\rightarrow$


$-$


\begin{flushleft}
$\partial$x $\partial$y
\end{flushleft}


\begin{flushleft}
$\partial$x $\partial$x $\partial$y $\partial$y
\end{flushleft}


\begin{flushleft}
$\partial$x $\partial$y
\end{flushleft}


$-$


\begin{flushleft}
F0 = $\rightarrow$
\end{flushleft}


·


=


\begin{flushleft}
a · b =
\end{flushleft}


,


,


+


,


\begin{flushleft}
$\partial$u $\partial$u
\end{flushleft}


\begin{flushleft}
$\partial$v $\partial$v
\end{flushleft}


\begin{flushleft}
$\partial$u $\partial$v
\end{flushleft}


\begin{flushleft}
$\partial$u $\partial$v
\end{flushleft}


?


?


?


?


?


?


?


?


$\rightarrow$


$-$ $\rightarrow$


$-$


\begin{flushleft}
$\partial$x $\partial$y
\end{flushleft}


\begin{flushleft}
$\partial$x 2
\end{flushleft}


\begin{flushleft}
$\partial$y 2
\end{flushleft}


\begin{flushleft}
$\partial$x $\partial$y
\end{flushleft}


,


·


,


=


+


.


\begin{flushleft}
G0 = b · b =
\end{flushleft}


\begin{flushleft}
$\partial$v $\partial$v
\end{flushleft}


\begin{flushleft}
$\partial$v $\partial$v
\end{flushleft}


\begin{flushleft}
$\partial$v
\end{flushleft}


\begin{flushleft}
$\partial$v
\end{flushleft}





(18)





(19)





\begin{flushleft}
As we mentioned earlier, since we are dealing with orthogonal curves, f = F = F 0 = 0. Using this fact
\end{flushleft}


\begin{flushleft}
and Eqs. (17), (18) and (19), the following relation can be derived (see Section X Chapter 2 in [5]):
\end{flushleft}


\begin{flushleft}
? $\partial$u ?2
\end{flushleft}


\begin{flushleft}
? $\partial$v ?2
\end{flushleft}


\begin{flushleft}
? $\partial$u ?2
\end{flushleft}


\begin{flushleft}
? $\partial$v ?2
\end{flushleft}


\begin{flushleft}
E=
\end{flushleft}


\begin{flushleft}
E0 +
\end{flushleft}


\begin{flushleft}
G0 ,
\end{flushleft}


\begin{flushleft}
G=
\end{flushleft}


\begin{flushleft}
E0 +
\end{flushleft}


\begin{flushleft}
G0 .
\end{flushleft}


(20)


\begin{flushleft}
$\partial$$\phi$
\end{flushleft}


\begin{flushleft}
$\partial$$\phi$
\end{flushleft}


\begin{flushleft}
$\partial$$\lambda$
\end{flushleft}


\begin{flushleft}
$\partial$$\lambda$
\end{flushleft}


\begin{flushleft}
From Eq. (13), a mapping from the Earth to the plotting surface requires that
\end{flushleft}


\begin{flushleft}
eg = EG.
\end{flushleft}


\begin{flushleft}
From Eqs. (18), (19), (20) and using F = F 0 = 0, one obtains
\end{flushleft}


\newpage



\begin{flushleft}
\newpage
 $\partial$u
\end{flushleft}


\newpage



\begin{flushleft}
\newpage
 $\partial$$\phi$
\end{flushleft}


\newpage



\begin{flushleft}
EG = J 2 · E 0 G0 , J = \newpage

\end{flushleft}


\begin{flushleft}
\newpage
 $\partial$v
\end{flushleft}


\newpage



\newpage



\begin{flushleft}
$\partial$$\phi$
\end{flushleft}





(21)





\newpage



\begin{flushleft}
$\partial$u \newpage

\end{flushleft}


\newpage



\begin{flushleft}
$\partial$$\lambda$ \newpage
\newpage

\end{flushleft}


\newpage
,


\begin{flushleft}
$\partial$v \newpage
\newpage

\end{flushleft}


\newpage



\begin{flushleft}
$\partial$$\lambda$
\end{flushleft}





(22)





\begin{flushleft}
where J is the Jacobian determinant of the transformation from the coordinate set $\phi$ and $\lambda$ to the
\end{flushleft}


\begin{flushleft}
coordinate set u and v.
\end{flushleft}


\begin{flushleft}
By a theorem of differential geometry (see [5]), a mapping for the orthogonal curves is conformal if
\end{flushleft}


\begin{flushleft}
and only if
\end{flushleft}


\begin{flushleft}
E
\end{flushleft}


\begin{flushleft}
G
\end{flushleft}


= .


\begin{flushleft}
e
\end{flushleft}


\begin{flushleft}
g
\end{flushleft}





4





(23)





\begin{flushleft}
Projection from an ellipsoid to a sphere
\end{flushleft}





\begin{flushleft}
In this section, we describe how much the latitudes and longitudes of a spheroidal model of the Earth
\end{flushleft}


\begin{flushleft}
will be effected once they are transformed to a spherical model, i.e., how much distortion in shape and
\end{flushleft}


\begin{flushleft}
size happens when one projects a spheroidal model of the Earth to a spherical model, [2, 5, 6, 8]. We
\end{flushleft}


\begin{flushleft}
distinguish two cases, equal-area transformation and conformal transformation.
\end{flushleft}


\begin{flushleft}
Case 1. A spherical model of the Earth that has the same surface area as that of the reference ellipsoid
\end{flushleft}


\begin{flushleft}
is called the authalic sphere. This sphere may be used as an intermediate step in the transformation from
\end{flushleft}


\begin{flushleft}
the ellipsoid to the mapping surface.
\end{flushleft}


8





\begin{flushleft}
\newpage
Let RA , $\phi$A and $\lambda$A be the authalic radius, latitude and longitude, respectively. Also, let $\phi$ and $\lambda$
\end{flushleft}


\begin{flushleft}
be the geodetic latitude and longitude, respectively. From Example 1, we have e = M 2 , g = N 2 cos2 $\phi$,
\end{flushleft}


2


2


\begin{flushleft}
E 0 = RA
\end{flushleft}


\begin{flushleft}
and G0 = RA
\end{flushleft}


\begin{flushleft}
cos2 $\phi$. By Eqs. (21) and (22),
\end{flushleft}


\newpage



\begin{flushleft}
\newpage
 $\partial$$\phi$A
\end{flushleft}


\newpage



\begin{flushleft}
\newpage
 $\partial$$\phi$
\end{flushleft}


\newpage



4


\begin{flushleft}
M 2 N 2 cos2 $\phi$ = RA
\end{flushleft}


\begin{flushleft}
cos2 $\phi$A \newpage

\end{flushleft}


\begin{flushleft}
\newpage
 $\partial$$\lambda$
\end{flushleft}


\begin{flushleft}
\newpage
 A
\end{flushleft}


\newpage



\begin{flushleft}
$\partial$$\phi$
\end{flushleft}





\newpage



\begin{flushleft}
$\partial$$\phi$A \newpage
2
\end{flushleft}


\newpage



\begin{flushleft}
$\partial$$\lambda$ \newpage
\newpage

\end{flushleft}


\newpage
 .


\begin{flushleft}
$\partial$$\lambda$A \newpage
\newpage

\end{flushleft}


\newpage



\begin{flushleft}
$\partial$$\lambda$
\end{flushleft}





(24)





\begin{flushleft}
In the transformation from the ellipsoid to the authalic sphere, longitude is invariant, i.e., $\lambda$ = $\lambda$A .
\end{flushleft}


\begin{flushleft}
Moreover, $\phi$A is independent of $\lambda$A and so $\lambda$. Thus Eq. (24) reduces to
\end{flushleft}


\begin{flushleft}
\newpage
 $\partial$$\phi$
\end{flushleft}


\begin{flushleft}
\newpage
 A
\end{flushleft}


\newpage



\begin{flushleft}
\newpage
 $\partial$$\phi$
\end{flushleft}


2 2


2


4


2


\begin{flushleft}
M N cos $\phi$ = RA cos $\phi$A \newpage

\end{flushleft}


\newpage



\newpage
 0





\newpage
2


\newpage



0\newpage



\newpage



\newpage
 .


\newpage



1\newpage






(25)





\begin{flushleft}
Substitute the values of M and N (given in Example 1) into Eq. (25) to obtain
\end{flushleft}


\begin{flushleft}
a2 (1 $-$ e2ab )
\end{flushleft}


2


\begin{flushleft}
cos $\phi$d$\phi$ = RA
\end{flushleft}


\begin{flushleft}
cos $\phi$A d$\phi$A .
\end{flushleft}


\begin{flushleft}
(1 $-$ e2ab sin2 $\phi$)2
\end{flushleft}





(26)





\begin{flushleft}
Integrating the left hand side of Eq. (26) from 0 to $\phi$ (using binary expansion), and the right hand side
\end{flushleft}


\begin{flushleft}
from 0 to $\phi$A , one obtains
\end{flushleft}


?


?


2


3


4


2


\begin{flushleft}
RA
\end{flushleft}


\begin{flushleft}
sin $\phi$A = a2 (1 $-$ e2ab ) sin $\phi$ + e2ab sin3 $\phi$ + e4ab sin5 $\phi$ + e6ab sin7 $\phi$ + · · · .
\end{flushleft}


3


5


7





(27)





\begin{flushleft}
Assuming $\phi$A = $\pi$/2 when $\phi$ = $\pi$/2, Eq. (27) gives:
\end{flushleft}


?


?


2


3


4


\begin{flushleft}
RA = a2 (1 $-$ e2ab ) 1 + e2ab + e4ab + e6ab + · · · .
\end{flushleft}


3


5


7





(28)





\begin{flushleft}
Substituting Eq. (28) into Eq. (27), one obtains
\end{flushleft}


?


\begin{flushleft}
sin $\phi$A = sin $\phi$
\end{flushleft}





?


\begin{flushleft}
1 + 23 e2ab sin2 $\phi$ + 53 e4ab sin4 $\phi$ + 47 e6ab sin6 $\phi$ + · · ·
\end{flushleft}


.


\begin{flushleft}
1 + 23 e2ab + 35 e4ab + 47 e6ab + · · ·
\end{flushleft}





(29)





\begin{flushleft}
Since the eccentricity eab is a small number, the above series are convergent. The relation between
\end{flushleft}


\begin{flushleft}
authalic and geodetic latitudes is equal at latitudes 0◦ and 90◦ , and the difference between them at other
\end{flushleft}


\begin{flushleft}
latitudes is about 0◦ .1 for the WGS-84 spheroid (see [5] for the definitions of the WGS-84 and WGS-72
\end{flushleft}


\begin{flushleft}
spheroids).
\end{flushleft}


\begin{flushleft}
Example 2
\end{flushleft}


\begin{flushleft}
1. For the WGS-72 spheroid with a $\approx$ 6, 378, 135 m and eab $\approx$ 0.081818, the radius of the authalic
\end{flushleft}


\begin{flushleft}
sphere is
\end{flushleft}


\begin{flushleft}
r
\end{flushleft}


?


2


3 ?


\begin{flushleft}
RA $\approx$ a (1 $-$ e2ab ) 1 + e2ab + e4ab $\approx$ 6, 371, 004 m.
\end{flushleft}


3


5


\begin{flushleft}
2. For the I.U.G.G spheroid with f = (a $-$ b)/a $\approx$ 1/298.275, we have eab = 2f $-$ f 2 $\approx$ 0.0066944, and
\end{flushleft}


\begin{flushleft}
from Eq. (29), for geodetic latitude $\phi$ = 45◦ , we have sin $\phi$A $\approx$ 0.70552 which gives $\phi$A $\approx$ 44◦ .8713.
\end{flushleft}





9





\begin{flushleft}
\newpage
Case 2. A conformal sphere is an sphere defined for conformal transformation from an ellipsoid,
\end{flushleft}


\begin{flushleft}
and similar to the authalic sphere may be used as an intermediate step in the transformation from the
\end{flushleft}


\begin{flushleft}
reference ellipsoid to a mapping surface.
\end{flushleft}


\begin{flushleft}
Let Rc , $\phi$c and $\lambda$c be the conformal radius, latitude and longitude for the conformal sphere, respectively. Let e and g be the same fundamental quantities as Case 1, and E 0 = Rc2 and G0 = Rc2 cos2 $\phi$c .
\end{flushleft}


\begin{flushleft}
Also, let $\phi$c = $\phi$c ($\phi$) and $\lambda$c = $\lambda$. Thus, from Eq. (20),
\end{flushleft}


\begin{flushleft}
E=
\end{flushleft}





\begin{flushleft}
? $\partial$$\phi$ ?2
\end{flushleft}


\begin{flushleft}
c
\end{flushleft}





\begin{flushleft}
$\partial$$\phi$
\end{flushleft}





\begin{flushleft}
E0 +
\end{flushleft}





\begin{flushleft}
? $\partial$$\lambda$ ?2
\end{flushleft}


\begin{flushleft}
c
\end{flushleft}





\begin{flushleft}
$\partial$$\phi$
\end{flushleft}





\begin{flushleft}
G0 =
\end{flushleft}





\begin{flushleft}
? $\partial$$\phi$ ?2
\end{flushleft}


\begin{flushleft}
c
\end{flushleft}





\begin{flushleft}
$\partial$$\phi$
\end{flushleft}





\begin{flushleft}
E0,
\end{flushleft}





\begin{flushleft}
G=
\end{flushleft}





\begin{flushleft}
? $\partial$$\phi$ ?2
\end{flushleft}


\begin{flushleft}
c
\end{flushleft}





\begin{flushleft}
$\partial$$\lambda$
\end{flushleft}





\begin{flushleft}
E0 +
\end{flushleft}





\begin{flushleft}
? $\partial$$\lambda$ ?2
\end{flushleft}


\begin{flushleft}
c
\end{flushleft}





\begin{flushleft}
$\partial$$\lambda$
\end{flushleft}





\begin{flushleft}
G0 = G0 .
\end{flushleft}





(30)





\begin{flushleft}
Combining Eqs. (23) and (30), one obtains
\end{flushleft}


\begin{flushleft}
? $\partial$$\phi$ ?2
\end{flushleft}


\begin{flushleft}
c
\end{flushleft}





\begin{flushleft}
$\partial$$\phi$
\end{flushleft}


\begin{flushleft}
M2
\end{flushleft}





\begin{flushleft}
Rc2
\end{flushleft}


=





\begin{flushleft}
Rc2 cos2 $\phi$c
\end{flushleft}


,


\begin{flushleft}
N 2 cos2 $\phi$
\end{flushleft}





\begin{flushleft}
that after integrating and simplifying with the condition $\phi$c = 0 for $\phi$ = 0, it gives
\end{flushleft}





\begin{flushleft}
tan
\end{flushleft}





\begin{flushleft}
eab
\end{flushleft}


\begin{flushleft}
? $\phi$ $\pi$ ?? 1 $-$ e sin $\phi$ ?
\end{flushleft}


\begin{flushleft}
$\pi$?
\end{flushleft}


\begin{flushleft}
ab
\end{flushleft}


2


.


+


\begin{flushleft}
= tan
\end{flushleft}


+


2


4


2


4


\begin{flushleft}
1 + eab sin $\phi$
\end{flushleft}





\begin{flushleft}
?$\phi$
\end{flushleft}





\begin{flushleft}
c
\end{flushleft}





(31)





\begin{flushleft}
One can
\end{flushleft}


\begin{flushleft}
$\surd$ calculate $\phi$c from Eq. (31) which is a function of geodetic latitude $\phi$. Also, it can be shown that
\end{flushleft}


\begin{flushleft}
Rc = M N for a given latitude $\phi$ which in this case $\phi$ = $\pi$/2. We refer to Chapter 5 Section 3 in [5] for
\end{flushleft}


\begin{flushleft}
the derivation.
\end{flushleft}





5





\begin{flushleft}
Albers and Lambert, one standard parallel
\end{flushleft}





\begin{flushleft}
In this section, we describe the Albers one standard parallel (equal-area conic projection) and Lambert
\end{flushleft}


\begin{flushleft}
one standard parallel (conformal conic projection) at latitude $\phi$0 which give good maps around that
\end{flushleft}


\begin{flushleft}
latitude (cf., [1, 5, 6, 8]).
\end{flushleft}


\begin{flushleft}
We start with some geometric properties in a cone tangent to a spherical model of the Earth at
\end{flushleft}


\begin{flushleft}
latitude $\phi$0 . In Fig. 6, ACN and BDN are two meridians separated by a longitude difference of ∆$\lambda$, and
\end{flushleft}


\begin{flushleft}
CD is an arc of the circle parallel to the equator. We have CD = DO0 ∆$\lambda$ and DN 0 sin $\phi$0 = DO0 and
\end{flushleft}


\begin{flushleft}
approximately $\theta$ · DN 0 = CD.
\end{flushleft}


\begin{flushleft}
Therefore, the first polar coordinate, $\theta$, is a linear function of $\lambda$, i.e.,
\end{flushleft}


\begin{flushleft}
$\theta$ = ∆$\lambda$ sin $\phi$0 .
\end{flushleft}





(32)





\begin{flushleft}
The second polar coordinate, r, is a function of $\phi$, i.e.,
\end{flushleft}


\begin{flushleft}
r = r($\phi$).
\end{flushleft}





(33)





\begin{flushleft}
The constant of the cone, denoted \%, is defined from the relation between lengths on the developed cone
\end{flushleft}


\begin{flushleft}
on the Earth. Let the total angle on the cone, $\theta$T , corresponding to 2$\pi$ on the Earth be $\theta$T = d/r0 , where
\end{flushleft}


\begin{flushleft}
d = 2$\pi$R cos $\phi$0 is the circumference of the parallel circle to the equator at latitude $\phi$0 , and r0 = R cot $\phi$0 .
\end{flushleft}


\begin{flushleft}
Thus $\theta$T = 2$\pi$ sin $\phi$0 , and the constant of the cone is defined as \% = sin $\phi$0 .
\end{flushleft}


\begin{flushleft}
Case 1. The Albers projection. Consider a spherical model of the Earth. From Example 1, we know
\end{flushleft}


\begin{flushleft}
that the first fundamental quantities for the sphere are e = R2 and g = R2 cos2 $\phi$ and for a cone (the
\end{flushleft}





10





\begin{flushleft}
\newpage
Figure 6: Geometry for angular convergence of the meridians
\end{flushleft}


\begin{flushleft}
polar coordinate system) are E 0 = 1 and G0 = r2 . Hence, from Eqs. (21) and (22),
\end{flushleft}


\newpage



\begin{flushleft}
\newpage
 $\partial$r
\end{flushleft}


\newpage



\begin{flushleft}
\newpage
 $\partial$$\phi$
\end{flushleft}


4


2


2\newpage



\begin{flushleft}
R cos $\phi$ = r \newpage

\end{flushleft}


\begin{flushleft}
\newpage
 $\partial$$\theta$
\end{flushleft}


\newpage



\newpage



\begin{flushleft}
$\partial$$\phi$
\end{flushleft}





\newpage



\begin{flushleft}
$\partial$r \newpage
2
\end{flushleft}


\newpage



\begin{flushleft}
$\partial$$\lambda$ \newpage
\newpage

\end{flushleft}


\newpage
 .


\begin{flushleft}
$\partial$$\theta$ \newpage
\newpage

\end{flushleft}


\newpage



\begin{flushleft}
$\partial$$\lambda$
\end{flushleft}





(34)





\newpage
2


\newpage



0 \newpage



\newpage



\newpage
 .


\newpage



\begin{flushleft}
sin $\phi$ \newpage

\end{flushleft}





(35)





\begin{flushleft}
Using Eqs. (32) and (33), Eq. (34) becomes
\end{flushleft}


\begin{flushleft}
\newpage
 $\partial$r
\end{flushleft}


\newpage



\newpage



\begin{flushleft}
\newpage
 $\partial$$\phi$
\end{flushleft}


\begin{flushleft}
R4 cos2 $\phi$ = r2 \newpage

\end{flushleft}


\newpage



\newpage
 0





0





\begin{flushleft}
Solving Eq. (35) by knowing the fact that an increase in $\phi$ corresponds to a decrease in r, one gets
\end{flushleft}


\begin{flushleft}
r2 =
\end{flushleft}





\begin{flushleft}
$-$2R2 sin $\phi$
\end{flushleft}


\begin{flushleft}
+ c.
\end{flushleft}


\begin{flushleft}
sin $\phi$0
\end{flushleft}





(36)





\begin{flushleft}
Imposing the boundary condition r0 = R cot $\phi$0 into Eq. (36), c = 2R2 + R2 cot2 $\phi$0 , and so after some
\end{flushleft}


\begin{flushleft}
simplifications, Eq. (36) becomes
\end{flushleft}


\begin{flushleft}
q
\end{flushleft}


\begin{flushleft}
R
\end{flushleft}


\begin{flushleft}
1 + sin2 $\phi$0 $-$ 2 sin $\phi$ sin $\phi$0 .
\end{flushleft}


(37)


\begin{flushleft}
r=
\end{flushleft}


\begin{flushleft}
sin $\phi$0
\end{flushleft}


\begin{flushleft}
The Cartesian plotting equations for a conical projection are defined as follows:
\end{flushleft}


\begin{flushleft}
x = sr sin $\theta$,
\end{flushleft}





\begin{flushleft}
y = s(r0 $-$ r cos $\theta$),
\end{flushleft}





11





(38)





\begin{flushleft}
\newpage
where s is the scale factor, $\theta$ and r are given respectively by Eqs. (32) and (37), and r0 = R cot $\phi$0 . The
\end{flushleft}


\begin{flushleft}
origin of the projection has the coordinates $\lambda$0 (the longitude of central meridian) and $\phi$0 . Fig. 7 shows
\end{flushleft}


\begin{flushleft}
the Albers projection with one standard parallel.
\end{flushleft}


\begin{flushleft}
If we let $\phi$0 = 90◦ , then Eqs. (32) and (37) reduce to
\end{flushleft}


\begin{flushleft}
p
\end{flushleft}


\begin{flushleft}
$\theta$ = ∆$\lambda$,
\end{flushleft}


\begin{flushleft}
r = R 2(1 $-$ sin $\phi$),
\end{flushleft}


\begin{flushleft}
that are the polar coordinates for the azimuthal equal-area projection, a special case of the Albers
\end{flushleft}


\begin{flushleft}
projection, see Fig. 8.
\end{flushleft}





\begin{flushleft}
Figure 8: The Albers azimuthal map
\end{flushleft}


\begin{flushleft}
Figure 7: The Albers equal-area map with standard
\end{flushleft}


\begin{flushleft}
parallel 45◦ N.
\end{flushleft}


\begin{flushleft}
Case 2. The Lambert projection. In this case, we consider a spheroidal model of the Earth. From
\end{flushleft}


\begin{flushleft}
Example 1, the fundamental quantities for this model are e = M 2 and g = N 2 cos2 $\phi$, and the fundamental
\end{flushleft}


\begin{flushleft}
quantities for a cone are E 0 = 1 and G0 = r2 . Again using Eqs. (32) and (33), Eq. (20) becomes
\end{flushleft}


\begin{flushleft}
E=
\end{flushleft}





\begin{flushleft}
? $\partial$r ?2
\end{flushleft}


\begin{flushleft}
$\partial$$\phi$
\end{flushleft}





\begin{flushleft}
E0 +
\end{flushleft}





\begin{flushleft}
? $\partial$$\theta$ ?2
\end{flushleft}


\begin{flushleft}
$\partial$$\phi$
\end{flushleft}





\begin{flushleft}
G0 =
\end{flushleft}





\begin{flushleft}
? $\partial$r ?2
\end{flushleft}


\begin{flushleft}
$\partial$$\phi$
\end{flushleft}





,





\begin{flushleft}
G=
\end{flushleft}





\begin{flushleft}
? $\partial$r ?2
\end{flushleft}


\begin{flushleft}
$\partial$$\lambda$
\end{flushleft}





\begin{flushleft}
E0 +
\end{flushleft}





\begin{flushleft}
? $\partial$$\theta$ ?2
\end{flushleft}


\begin{flushleft}
$\partial$$\lambda$
\end{flushleft}





\begin{flushleft}
G0 = sin2 $\phi$0 r2 .
\end{flushleft}





(39)





\begin{flushleft}
Substituting these values in Eq. (23), integrating, simplifying and noting that r increases as $\phi$ decreases,
\end{flushleft}


\begin{flushleft}
one gets
\end{flushleft}


\begin{flushleft}

\end{flushleft}


\begin{flushleft}
sin $\phi$0
\end{flushleft}


\begin{flushleft}
? $\pi$ $\phi$ ?? 1 + e sin $\phi$ ?eab /2 
\end{flushleft}


\begin{flushleft}

\end{flushleft}


\begin{flushleft}

\end{flushleft}


\begin{flushleft}
ab
\end{flushleft}


\begin{flushleft}

\end{flushleft}


\begin{flushleft}

\end{flushleft}


\begin{flushleft}

\end{flushleft}


\begin{flushleft}

\end{flushleft}


\begin{flushleft}

\end{flushleft}


\begin{flushleft}
 tan 4 $-$ 2
\end{flushleft}


\begin{flushleft}
1 $-$ eab sin $\phi$
\end{flushleft}


,


\begin{flushleft}
r = r0
\end{flushleft}


\begin{flushleft}
? $\pi$ $\phi$ ?? 1 + e sin $\phi$ ?eab /2 
\end{flushleft}


\begin{flushleft}

\end{flushleft}


\begin{flushleft}

\end{flushleft}


\begin{flushleft}

\end{flushleft}


0


\begin{flushleft}
ab
\end{flushleft}


0


\begin{flushleft}

\end{flushleft}


\begin{flushleft}

\end{flushleft}


\begin{flushleft}

\end{flushleft}


\begin{flushleft}
 tan
\end{flushleft}


$-$


4


2


\begin{flushleft}
1 $-$ eab sin $\phi$0
\end{flushleft}


\begin{flushleft}
where
\end{flushleft}


\begin{flushleft}
r0 = N$\phi$0 cot $\phi$0 =
\end{flushleft}





\begin{flushleft}
a cot $\phi$0
\end{flushleft}


\begin{flushleft}
1 $-$ e2ab sin2 $\phi$0
\end{flushleft}





(40)





?0.5 .





\begin{flushleft}
The Cartesian equations are the same as Eq. (38) with these new r0 and r. Fig. 9 shows the Lambert
\end{flushleft}


\begin{flushleft}
projection with one standard parallel.
\end{flushleft}





12





\begin{flushleft}
\newpage
Figure 9: The Lambert conformal map
\end{flushleft}





6





\begin{flushleft}
Sinusoidal projection
\end{flushleft}





\begin{flushleft}
In this section, we only discuss about the sinusoidal equal-area projection that is a projection of the entire
\end{flushleft}


\begin{flushleft}
model of the Earth onto a single map, and it gives an adequate whole world coverage, [2, 5].
\end{flushleft}


\begin{flushleft}
Consider a spherical model of the Earth with the fundamental quantities e = R2 and g = R2 cos2 $\phi$.
\end{flushleft}


\begin{flushleft}
The first fundamental quantities on a planar mapping surface is E 0 = G0 = 1. Substituting these
\end{flushleft}


\begin{flushleft}
fundamental quantities into Eq. (22) (using Eq. (21)), one gets
\end{flushleft}


\newpage



\newpage
2


\begin{flushleft}
\newpage
 $\partial$x $\partial$x \newpage

\end{flushleft}


\newpage



\newpage



\begin{flushleft}
\newpage
 $\partial$$\phi$ $\partial$$\lambda$ \newpage

\end{flushleft}


\newpage



\newpage



\begin{flushleft}
R4 cos2 $\phi$ = \newpage

\end{flushleft}


\newpage
 ,


\begin{flushleft}
\newpage
 $\partial$y $\partial$y \newpage

\end{flushleft}


\newpage



\newpage



\newpage



\newpage



\begin{flushleft}
$\partial$$\phi$ $\partial$$\lambda$
\end{flushleft}


\begin{flushleft}
which by imposing the conditions y = R$\phi$ and x = x($\phi$, $\lambda$) reduces to
\end{flushleft}


\begin{flushleft}
\newpage
 $\partial$x $\partial$x \newpage
2
\end{flushleft}


\newpage



\newpage



\newpage



\newpage



\begin{flushleft}
? $\partial$x ?2
\end{flushleft}


\newpage



\newpage



\begin{flushleft}
$\partial$$\phi$
\end{flushleft}


\begin{flushleft}
$\partial$$\lambda$
\end{flushleft}


\begin{flushleft}
R4 cos2 $\phi$ = \newpage

\end{flushleft}


.


\begin{flushleft}
\newpage
 = R2
\end{flushleft}


\newpage



\newpage



\begin{flushleft}
$\partial$$\lambda$
\end{flushleft}


\begin{flushleft}
\newpage
R
\end{flushleft}


\newpage



0





(41)





\begin{flushleft}
Taking the positive square root of Eq. (41) and using the fact that $\lambda$ and $\phi$ are independent, one obtains
\end{flushleft}


\begin{flushleft}
dx = R cos $\phi$d$\lambda$, and so by integrating x = $\lambda$R cos $\phi$ + c. Using the boundary condition x = 0 when
\end{flushleft}


\begin{flushleft}
$\lambda$ = $\lambda$0 , one gets c = $-$$\lambda$0 R cos $\phi$, and so the plotting equations for the sinusoidal projection become as
\end{flushleft}


\begin{flushleft}
follow ($\phi$ and $\lambda$ in radians):
\end{flushleft}


\begin{flushleft}
x = sR∆$\lambda$ cos $\phi$,
\end{flushleft}





\begin{flushleft}
y = sR$\phi$,
\end{flushleft}





(42)





\begin{flushleft}
where s is the scale factor. Fig. 10 shows a normalized plot for the sinusoidal projection. In this map,
\end{flushleft}


\begin{flushleft}
the meridians are sinusoidal curves except the central meridian which is a vertical line and they all meet
\end{flushleft}


\begin{flushleft}
each other in the poles. This is why this map is known as the sinusoidal map. The x axis is also along
\end{flushleft}


\begin{flushleft}
the equator.
\end{flushleft}


\begin{flushleft}
The inverse transformation from the Cartesian to geographic coordinates is simply calculated from
\end{flushleft}


\begin{flushleft}
Eq. (42):
\end{flushleft}


\begin{flushleft}
x
\end{flushleft}


\begin{flushleft}
y
\end{flushleft}


,


\begin{flushleft}
∆$\lambda$ =
\end{flushleft}


.


\begin{flushleft}
$\phi$=
\end{flushleft}


\begin{flushleft}
sR
\end{flushleft}


\begin{flushleft}
sR cos $\phi$
\end{flushleft}


13





\begin{flushleft}
\newpage
Figure 10: The sinusoidal equal-area projection with Tissot's indicatrices that are changing their shape
\end{flushleft}


\begin{flushleft}
(the ellipses with different eccentricities indicating angular distortion) toward the poles while having the
\end{flushleft}


\begin{flushleft}
same size.
\end{flushleft}





7





\begin{flushleft}
Some conventional projections
\end{flushleft}





\begin{flushleft}
In this section, we give the plotting equations for two conventional projections, the simple conic projection
\end{flushleft}


\begin{flushleft}
(one standard parallel) and the plate carree projection (cf., [2, 5, 7]). As we mentioned earlier, these
\end{flushleft}


\begin{flushleft}
projections neither preserve the shape nor do they preserve the size, and they are usually used for simple
\end{flushleft}


\begin{flushleft}
portrayals of the world or regions with minimal geographic data such as index maps.
\end{flushleft}


\begin{flushleft}
1. The simple conic projection is a projection that the distances along every meridian are true scale.
\end{flushleft}


\begin{flushleft}
Suppose that the conic is tangent to the spherical model of the Earth at latitude $\phi$0 , see Fig. 11. In
\end{flushleft}


\begin{flushleft}
this figure, we have r0 = R cot $\phi$0 . We want to have DE = DE 0 , but DE 0 = R($\phi$ $-$ $\phi$0 ). Thus the polar
\end{flushleft}


\begin{flushleft}
coordinates for this projection are
\end{flushleft}


\begin{flushleft}
r = r0 $-$ R($\phi$ $-$ $\phi$0 ),
\end{flushleft}





\begin{flushleft}
$\theta$ = ∆$\lambda$ sin $\phi$0 .
\end{flushleft}





\begin{flushleft}
Replacing these values into Eq. (38) gives its Cartesian coordinates.
\end{flushleft}





\begin{flushleft}
Figure 11: Geometry for the simple conic projection
\end{flushleft}





14





\begin{flushleft}
\newpage
2. The plate carree, the equirectangular projection, is a conventional cylindrical projection that
\end{flushleft}


\begin{flushleft}
divides the meridians equally the same way as on the sphere. Also, it divides the equator and its parallels
\end{flushleft}


\begin{flushleft}
equally. The plate carree plotting equations are very simple:
\end{flushleft}


\begin{flushleft}
x = sR∆$\lambda$,
\end{flushleft}





\begin{flushleft}
y = sR$\phi$,
\end{flushleft}





\begin{flushleft}
where $\phi$ and $\lambda$ are in radians. Fig. 12 shows the plate carree map with Tissot's indicatrices which are
\end{flushleft}


\begin{flushleft}
changing their shape and size when moving toward the poles indicating that this map is neither equal-area
\end{flushleft}


\begin{flushleft}
nor conformal.
\end{flushleft}





\begin{flushleft}
Figure 12: The plate carree map, 10◦ graticule.
\end{flushleft}





8





\begin{flushleft}
Theory of distortion
\end{flushleft}





\begin{flushleft}
In this section, we discuss about three types of distortions from differential geometry approach: distortions
\end{flushleft}


\begin{flushleft}
in length, area and angle, and we present them in term of the Gaussian fundamental quantities (cf.,
\end{flushleft}


[5, 6, 9]).


\begin{flushleft}
1. The distortion in length is defined as the ratio of a length of a line on a map to the length of the
\end{flushleft}


\begin{flushleft}
true line on a model of the Earth. More precisely,
\end{flushleft}


\begin{flushleft}
KL2 =
\end{flushleft}





\begin{flushleft}
(ds)2M
\end{flushleft}


\begin{flushleft}
E(d$\phi$)2 + G(d$\lambda$)2
\end{flushleft}


=


.


\begin{flushleft}
(ds)2E
\end{flushleft}


\begin{flushleft}
e(d$\phi$)2 + g(d$\lambda$)2
\end{flushleft}


\begin{flushleft}
r
\end{flushleft}





\begin{flushleft}
From Eq. (43), the distortion along the meridians (d$\lambda$ = 0) is Km =
\end{flushleft}


\begin{flushleft}
r
\end{flushleft}


\begin{flushleft}
G
\end{flushleft}


.


\begin{flushleft}
the equator (d$\phi$ = 0) is Ke =
\end{flushleft}


\begin{flushleft}
g
\end{flushleft}





(43)


\begin{flushleft}
E
\end{flushleft}


\begin{flushleft}
, and along the lines parallel to
\end{flushleft}


\begin{flushleft}
e
\end{flushleft}





\begin{flushleft}
2. The distortion in area is defined as the ratio of an area on a map
\end{flushleft}


\begin{flushleft}
$\surd$ to the true area on a model of
\end{flushleft}


\begin{flushleft}
the Earth. From Eq. (13) (f = F = 0), the area on the map is AM = EG, and the corresponding area
\end{flushleft}


$\surd$


\begin{flushleft}
on the model of the Earth is AE = eg. Thus, the distortion in area is
\end{flushleft}


\begin{flushleft}
s
\end{flushleft}


\begin{flushleft}
AM
\end{flushleft}


\begin{flushleft}
EG
\end{flushleft}


\begin{flushleft}
KA =
\end{flushleft}


=


\begin{flushleft}
= Km Ke .
\end{flushleft}


(44)


\begin{flushleft}
AE
\end{flushleft}


\begin{flushleft}
eg
\end{flushleft}





15





\begin{flushleft}
\newpage
In equal-area map projections, from Eq. (21), KA = Km Ke = 1.
\end{flushleft}


\begin{flushleft}
3. The distortion in angle is defined as (in percentage):
\end{flushleft}


\begin{flushleft}
K$\alpha$ = 100 ·
\end{flushleft}





\begin{flushleft}
$\alpha$$-$$\beta$
\end{flushleft}


,


\begin{flushleft}
$\alpha$
\end{flushleft}





(45)





\begin{flushleft}
where $\alpha$ is the angle on a model of the Earth (the azimuth), and $\beta$ is the projected angle on a map (the
\end{flushleft}


\begin{flushleft}
azimuth $\alpha$ on the map, cf., Fig. 13).
\end{flushleft}





\begin{flushleft}
Figure 13: Geometry for differential parallelograms
\end{flushleft}


\begin{flushleft}
In order to obtain $\beta$ as a function of the fundamental quantities and $\alpha$, we first calculate sin($\beta$ $\pm$ $\alpha$).
\end{flushleft}


\begin{flushleft}
From Fig. 13, we have
\end{flushleft}


\begin{flushleft}
sin($\beta$ $\pm$ $\alpha$) = sin $\beta$ cos $\alpha$ $\pm$ cos $\beta$ sin $\alpha$
\end{flushleft}


?$\surd$


\begin{flushleft}
d$\lambda$ ??$\surd$ d$\phi$ ? ?$\surd$
\end{flushleft}


\begin{flushleft}
d$\phi$ ??$\surd$ d$\lambda$ ?
\end{flushleft}


=


\begin{flushleft}
G·
\end{flushleft}


\begin{flushleft}
e·
\end{flushleft}


$\pm$


\begin{flushleft}
E·
\end{flushleft}


\begin{flushleft}
g·
\end{flushleft}


\begin{flushleft}
dS
\end{flushleft}


\begin{flushleft}
ds
\end{flushleft}


\begin{flushleft}
dS
\end{flushleft}


\begin{flushleft}
ds
\end{flushleft}


\begin{flushleft}
d$\phi$ d$\lambda$ $\surd$
\end{flushleft}


\begin{flushleft}
= (Ke $\pm$ Km )
\end{flushleft}


\begin{flushleft}
eg.
\end{flushleft}


\begin{flushleft}
dS dS
\end{flushleft}


\begin{flushleft}
Hence,
\end{flushleft}


\begin{flushleft}
sin($\beta$ $-$ $\alpha$) =
\end{flushleft}





\begin{flushleft}
Ke $-$ Km
\end{flushleft}


\begin{flushleft}
sin($\beta$ + $\alpha$).
\end{flushleft}


\begin{flushleft}
Ke + Km
\end{flushleft}





\begin{flushleft}
Define
\end{flushleft}


\begin{flushleft}
f ($\beta$) = sin($\beta$ $-$ $\alpha$) $-$
\end{flushleft}





\begin{flushleft}
Ke $-$ Km
\end{flushleft}


\begin{flushleft}
sin($\beta$ + $\alpha$).
\end{flushleft}


\begin{flushleft}
Ke + Km
\end{flushleft}





\begin{flushleft}
Now the goal is to find the roots of f . This can be done by Newton's iteration as follows:
\end{flushleft}


\begin{flushleft}
$\beta$n+1 = $\beta$n $-$
\end{flushleft}





16





\begin{flushleft}
f ($\beta$n )
\end{flushleft}


,


\begin{flushleft}
f 0 ($\beta$n )
\end{flushleft}





\begin{flushleft}
\newpage
where
\end{flushleft}





\begin{flushleft}
Ke $-$ Km
\end{flushleft}


\begin{flushleft}
cos($\beta$n + $\alpha$).
\end{flushleft}


\begin{flushleft}
Ke + Km
\end{flushleft}


\begin{flushleft}
The iteration is rapidly convergent by letting $\beta$0 = $\alpha$. In conformal mapping, from Eq. (23), Ke = Km ,
\end{flushleft}


\begin{flushleft}
and so the function f will have a unique solution ($\beta$ = $\alpha$).
\end{flushleft}


\begin{flushleft}
f 0 ($\beta$n ) = cos($\beta$n $-$ $\alpha$) $-$
\end{flushleft}





\begin{flushleft}
Example 3 In this example, we show the distortions in length in the Albers projection with one standard
\end{flushleft}


\begin{flushleft}
parallel. From Example 1, the first fundamental form for the map is
\end{flushleft}


\begin{flushleft}
(ds)2M = (dr)2 + r2 (d$\theta$)2 = E 0 (dr)2 + G0 (d$\theta$)2 ,
\end{flushleft}





(46)





\begin{flushleft}
and the first fundamental form for the spherical model of the Earth is
\end{flushleft}


\begin{flushleft}
(ds)2E = R2 (d$\phi$)2 + R2 cos2 $\phi$(d$\lambda$)2 = e(d$\phi$)2 + g(d$\lambda$)2 .
\end{flushleft}





(47)





\begin{flushleft}
Taking the derivatives of Eqs. (32) and (35), one obtains
\end{flushleft}


\begin{flushleft}
d$\theta$ = sin $\phi$0 d$\lambda$,
\end{flushleft}





\begin{flushleft}
dr = p
\end{flushleft}





\begin{flushleft}
$-$R cos $\phi$ d$\phi$
\end{flushleft}


\begin{flushleft}
1 + sin2 $\phi$0 $-$ 2 sin $\phi$ sin $\phi$0
\end{flushleft}





,





\begin{flushleft}
respectively. Substitute the above equations into Eq. (46) to get
\end{flushleft}


\begin{flushleft}
(ds)2M =
\end{flushleft}





\begin{flushleft}
R2 cos2 $\phi$
\end{flushleft}


\begin{flushleft}
(d$\phi$)2 + r2 sin2 $\phi$0 (d$\lambda$)2 = E(d$\phi$)2 + G(d$\lambda$)2 .
\end{flushleft}


\begin{flushleft}
1 + sin2 $\phi$0 $-$ 2 sin $\phi$ sin $\phi$0
\end{flushleft}





(48)





\begin{flushleft}
Substituting (47) and (48) in Eq. (43) gives the total length distortion. Also,
\end{flushleft}


\begin{flushleft}
s
\end{flushleft}


\begin{flushleft}
p
\end{flushleft}


\begin{flushleft}
r
\end{flushleft}


\begin{flushleft}
E
\end{flushleft}


\begin{flushleft}
G
\end{flushleft}


\begin{flushleft}
1 + sin2 $\phi$0 $-$ 2 sin $\phi$ sin $\phi$0
\end{flushleft}


\begin{flushleft}
cos $\phi$
\end{flushleft}


\begin{flushleft}
Km =
\end{flushleft}


\begin{flushleft}
=p
\end{flushleft}


,


\begin{flushleft}
K
\end{flushleft}


=


=


,


\begin{flushleft}
e
\end{flushleft}


\begin{flushleft}
e
\end{flushleft}


\begin{flushleft}
g
\end{flushleft}


\begin{flushleft}
cos $\phi$
\end{flushleft}


\begin{flushleft}
1 + sin2 $\phi$0 $-$ 2 sin $\phi$ sin $\phi$0
\end{flushleft}


\begin{flushleft}
which are functions of $\phi$. Clearly, Km Ke = 1.
\end{flushleft}


\begin{flushleft}
Example 4 In this example, we first use the first fundamental form to obtain the plotting equations
\end{flushleft}


\begin{flushleft}
for the Mercator projection, and then we show its length and area distortion. From Example 1, the first
\end{flushleft}


\begin{flushleft}
fundamental form for the cylindrical surface (the Cartesian coordinate system) is
\end{flushleft}


\begin{flushleft}
(ds)2M = (dy)2 + (dx)2 .
\end{flushleft}





(49)





\begin{flushleft}
Taking the derivative of Eq. (4) and substituting in Eq. (49), one finds
\end{flushleft}


\begin{flushleft}
? dy ?2
\end{flushleft}


\begin{flushleft}
(d$\phi$)2 + s2 R2 (d$\lambda$)2 = E(d$\phi$)2 + G(d$\lambda$)2 ,
\end{flushleft}


(50)


\begin{flushleft}
(ds)2M =
\end{flushleft}


\begin{flushleft}
d$\phi$
\end{flushleft}


?2


\begin{flushleft}
where s is the scale of the map along the equator, E = dy/d$\phi$ and G = s2 R2 . The first fundamental
\end{flushleft}


\begin{flushleft}
quantities for the spherical model of the Earth are e = R2 and g = R2 cos2 $\phi$. Substituting these
\end{flushleft}


\begin{flushleft}
fundamental quantities in Eq. (23) and simplifying, one obtains
\end{flushleft}


\begin{flushleft}
sRd$\phi$
\end{flushleft}


.


\begin{flushleft}
cos $\phi$
\end{flushleft}





\begin{flushleft}
dy =
\end{flushleft}





(51)





\begin{flushleft}
It is easy to see that integrating the above differential equation and applying the boundary condition
\end{flushleft}


?2


\begin{flushleft}
y(0) = 0, Eq. (7) follows. By Eq. (51), E = dy/d$\phi$ = s2 R2 / cos2 $\phi$. Therefore, substituting Eqs. (47)
\end{flushleft}


\begin{flushleft}
and (50) in Eq. (43), the length distortion will be
\end{flushleft}


\begin{flushleft}
KL =
\end{flushleft}





\begin{flushleft}
s
\end{flushleft}


.


\begin{flushleft}
cos $\phi$
\end{flushleft}





17





\begin{flushleft}
\newpage
It can be seen that KL = Km = Ke , and so from Eq. (44), the distortion in area for the Mercator
\end{flushleft}


\begin{flushleft}
projection is
\end{flushleft}


\begin{flushleft}
s2
\end{flushleft}


\begin{flushleft}
KA = Km Ke =
\end{flushleft}


.


\begin{flushleft}
cos2 $\phi$
\end{flushleft}


\begin{flushleft}
Hence, in the Mercator projection both length and area distortions are functions of $\phi$ not $\lambda$.
\end{flushleft}





9





\begin{flushleft}
Conclusion
\end{flushleft}





\begin{flushleft}
There are a number of map projections used for different purposes, and we discussed about three major
\end{flushleft}


\begin{flushleft}
classes of them, equal-area, conformal, and conventional. Users may also create their own map based on
\end{flushleft}


\begin{flushleft}
their projects by starting with a base map of known projection and scale.
\end{flushleft}


\begin{flushleft}
In this paper, in cylindrical projections, we assume that the cylinder is tangent to the equator. Making
\end{flushleft}


\begin{flushleft}
the cylinder tangent to other closed curves on the Earth results good maps in areas close to the tangency.
\end{flushleft}


\begin{flushleft}
This is also applied for conical and azimuthal projections.
\end{flushleft}


\begin{flushleft}
In all projections from a 3-D surface to a 2-D surface, there are distortions in length, shape or size that
\end{flushleft}


\begin{flushleft}
some of them can be removed (not all) or minimized from the map based on some specific applications.
\end{flushleft}


\begin{flushleft}
We also noticed in Section 4 that projecting a spheroidal model of the Earth to a spherical model of the
\end{flushleft}


\begin{flushleft}
Earth will also distort length, shape and angle.
\end{flushleft}


\begin{flushleft}
Intelligent map users should have knowledge about the theory of distortion in order to compare and
\end{flushleft}


\begin{flushleft}
distinguish their maps with the true surface on the Earth that they are studying.
\end{flushleft}





\begin{flushleft}
References
\end{flushleft}


\begin{flushleft}
[1] Davies, R. E. and Foote, F. S. and Kelly, J. E., Surveying: Theory and practice, McGraw-Hill, New
\end{flushleft}


\begin{flushleft}
York (1966)
\end{flushleft}


\begin{flushleft}
[2] Deetz, C. H. and Adams, O. S., Elements of map projection, Spec. Publ. 68, coast and geodetic
\end{flushleft}


\begin{flushleft}
survey, U. S. Gov't. Printing office, Washington, D. C. (1944)
\end{flushleft}


\begin{flushleft}
[3] Goetz, A., Introduction to differential geometry, Addison-Wesley, Reading, MA (1958)
\end{flushleft}


\begin{flushleft}
[4] Osserman, R., Mathematical Mapping from Mercator to the Millennium, Mathematical Sciences
\end{flushleft}


\begin{flushleft}
Research Institute (2004)
\end{flushleft}


\begin{flushleft}
[5] Pearson, F., Map projections, theory and applications, Boca Raton, Florida (1999)
\end{flushleft}


\begin{flushleft}
[6] Richardus, P. and Adler, R. K., Map projections for geodesists, cartographers, and geographers, North
\end{flushleft}


\begin{flushleft}
Holland, Amsterdam (1972)
\end{flushleft}


\begin{flushleft}
[7] Steers, J. A., An Introduction to the study of map projection, University of London (1962)
\end{flushleft}


\begin{flushleft}
[8] Thomas, P. D., Conformal projection in geodesy and cartography, Spec. Publ. 68, coast and geodetic
\end{flushleft}


\begin{flushleft}
survey, U. S. Gov't. Printing office, Washington, D. C. (1952)
\end{flushleft}


\begin{flushleft}
[9] Vanicek P. and Krakiwsky E. J., Geodesy the concepts, pp 697. Amsterdam The Netherlands, University of New Brunswick, Canada (1986)
\end{flushleft}





18





\newpage



\end{document}
